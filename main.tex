\documentclass[a4paper,12pt]{book}
\usepackage[utf8]{inputenc}
\usepackage{graphicx}
\usepackage{listings} %serve per importare il codice

\renewcommand\lstlistingname{Codice}

\usepackage{xcolor} %serve per scegliere i colori del codice che scrivi quelli che hai sono quelli che aveva il tizio su youtube
\definecolor{codegreen}{rgb}{0,0.6,0}
\definecolor{codegray}{rgb}{0.5,0.5,0.5}
\definecolor{codeorange}{rgb}{1,0.49,0}
\definecolor{backcolour}{rgb}{0.95,0.95,0.96}

\lstdefinestyle{mystyle}{
	backgroundcolor=\color{backcolour},   
	commentstyle=\color{codegray},
	keywordstyle=\color{codeorange},
	numberstyle=\tiny\color{codegray},
	stringstyle=\color{codegreen},
	basicstyle=\ttfamily\footnotesize,
	breakatwhitespace=false,         
	breaklines=true,                 
	captionpos=b,                    
	keepspaces=true,                 
	numbers=left,                    
	numbersep=5pt,                  
	showspaces=false,                
	showstringspaces=false,
	showtabs=false,                  
	tabsize=2,
	xleftmargin=10pt,
}

\lstset{style=mystyle}

\begin{document}

\author{nicolò Salimbeni}
\title{Guida di sopravvivenza a root}
\date{2021}

\mainmatter %serve per formattare le pagine (le numera con la 1 dall'inizio così coincidono con il pdf)
\maketitle  %prende le cose da autore titolo ecc.. e ci fa il titolo
\tableofcontents  %serve per l'indice all'inizio è una cosa che fa latex da solo in base a sezioni sottosezioni ecc...

\chapter{The First Chapter}
Prova di un codice qualunque di root con referenza \ref{cod1}:\\
\lstinputlisting[language=C++,caption={Codice},label={cod1},mathescape=true,breaklines=true]{./codici/codice.cxx}

\section{SEZIONE 1}
Come va? tutto apposto nella sezione 1?
\chapter{The Second Chapter}
\chapter{The Third Chapter}

\end{document}