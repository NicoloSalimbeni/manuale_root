\chapter{OGGETTI PRINCIPALI}
\section{TVectorD}
Sono gli equivalenti dei vettori di C++. Servono in alcune funzioni di root specifiche, come ad esempio per inserire i dati nei TGraph. funzionano come array, bisogna definirli con una dimensione come segue:
\begin{lstlisting}
	TVectorD x(n);
\end{lstlisting}
Dove n è un intero per la dimensione (se n=10 le posizioni sono da 0 a 9), ed x è il nome dell'oggetto. Vanno riempiti come gli array, ma ci sono molte funzioni che lavorano su di questi (vanno richiamate con l'operatore . ).
\begin{itemize}
	\item .GetNoElements() : restituisce un int con il numero di elementi
	\item .GetLwb() : "Get lower bound" ritorna la prima posizione, che è essenzialmente 0
	\item .GetUpb() : "Get upper bound" ritorna l'ultima posizione, quindi se ho 10 elementi ritorna 9
	\item .ResizeTo(int n) : ridimensiona il vettore, se n=10 allora il vettore dopo aver chiamato la funzione avrà 10 elementi, da 0 a 9.
\end{itemize}

\section{TH1}
\subsection{Costruttore}
Per creare un istogramma l'oggetto da utilizzare è il TH1D.
\begin{lstlisting}[language=C++,caption={Costruttore più comune per il TH1D},label={cod1},mathescape=true,breaklines=true]
	TH1D* h = new TH1D("name","title",nbin,begin,end)
\end{lstlisting}
\begin{itemize}
	\item "name" è il nome con cui verà chiamato all'interno della funzioni di root, per convenzione lo si chiama come l'oggetto, in questo caso quindi bisognerebbe scrivere name=h. In generale poichè dietro le quinte root usare il name per riferirsi a questo istogramma, non possono essere chiamati due istogrammi diversi con lo stesso nome, in tal caso il secondo sovrascrive il primo.
	\item "title" è il titotlo che viene visualizzato sul canvas quanto viene disegnato l'istogramma.
	\item "nbin" è il numero di bin (intero);
	\item begin e end (double) sono dove inizia e finisce l'istogramma, questo è anche il range che verrà stampato.
\end{itemize}
Questo non è l'unico costruttore possibile, si veda la documentazione per gli altri forniti.\\
{\color{red} \rule{\linewidth}{0.5mm}}
\textcolor{red}{\textbf{Attenzione:}} per convanzione i bin contengono il loro estramo \underline{inferiore}, di conseguenza se end=10 e nbin=200 allora il bin che contiene 10 sarà il 201, il quale è al di fuori dell'istogramma quindi anche al di fuori del range stampato.\\
{\color{red} \rule{\linewidth}{0.5mm}}

\subsection{Funzioni principali}
Vediamo ora le funzioni principali dell'oggetto:
\begin{lstlisting}[language=C++,label={cod1},mathescape=true,breaklines=true]
	h->Fill(3); //autmenta di un conteggio il bin che contiene 3, ritorna il numero del bin in questione;
	h->SetBinContent(455,25) //imposta il bin 455 a 25 conteggi, non ritorna nulla;
	h->Reset(); //elimina il contenuto dell'istogramma;
	h->Draw(); //stampa
	h->Draw("same"); //stampa in un canvas che contiene qualcosa
	h->FindBin(2); //ritorna il numero del bin che contiene 2
	h->GetMaximumBin(); //ritorna il bin contenente il massimo
	h->GetMaximum(); //ritorna i conteggi del bin piu alto
	h->GetBinContent(n); // ritorna il conteggio del bin n
	h->GetMean(); //media 
	h->GetMeanError(); //errore media  
	h->GetStdDev(); //deviazione standard
	h->GetStdDevError(); //errore sulla devstd
\end{lstlisting}
{\color{red} \rule{\linewidth}{0.5mm}}
\textcolor{red}{\textbf{Attenzione:}} tutte le funzioni che iniziano con Get lavorano nel range in cui l'istogramma è definito. Se però ad esempio durante l'analisi fai uno zoom sul canvas con il mouse lavorano nel range che vedi nel canvas (quello zoomato, NON tutto l'istogramma) quindi attento al range che utilizzi quando le chiami.\\
{\color{red} \rule{\linewidth}{0.5mm}}
Viste queste funzioni più basilari vediamone alcune più articolate:\\
\begin{lstlisting}[language=C++,label={cod1},mathescape=true,breaklines=true]
	h2->Add(h1,c1); // default c1=1
	h5->Add(h3,h4,c1,c2); // default c1=c2=1
\end{lstlisting}
La prima aggiunge i conteggi dell'istogramma h1 all'istogramma h2 con la seguente formula: h2 = h2 + c1* h1, questo permette in modo compatto di sommare sottrarre ecc... la seconda fa la somma degli istogrammi h3 ed h4 e la inserisce nel h5 come h5=h3*c1+h4*c2. Questo comando è utilizzabile anche se il numeri di bin dei vari istogrammi non è lo stesso, in tal caso viene chiamata una funzione apposita che cerca di aggiustare come può la situazione, di conseguenza cerca di evitarlo se possibile.\\
Una cosa da tenere in considerazione è che root in background mette in memoria delle informazioni riguardo i bin e gli errori dei bin dell'istogramma, tutto questo non ci deve interessare in quanto non vengono utilizzati direttamente dall'utente. Qualora però si voglia fare un fit è buona norma lanciare prima dei farlo il comando che segue
\begin{lstlisting}[language=C++,label={cod1},mathescape=true,breaklines=true]
	h->Sumw2();
\end{lstlisting}
per essere sicuri che il fit abbia tutte le informazioni corrette su cui basarsi. Tutto questo è descritto adeguatamente nella documentazione ufficiale dei TH1 (\href{https://root.cern.ch/doc/master/classTH1.html}{qui}) nella descrizione della funzione Add() dove sono specificati i casi in cui è necessario, in cui è facoltativo, e le possibili eccezioni.\\
Un'altra funzione utile nell'analisi dati è:
\begin{lstlisting}[language=C++,,label={cod1},mathescape=true,breaklines=true]
	h->Rebin(2)
\end{lstlisting}
Modifica il numero di bin dell'istogramma lasciando tutto il resto invariato. Nell'esempio 2 significa che due bin dell'istogramma prima della modifica diventeranno un solo bin dopo la modifica. In sostanza con 2 ne avrò la metà, con 3 un terzo ecc...
\begin{lstlisting}[language=C++,label={cod1},mathescape=true,breaklines=true]
	h->Integral();
	h->Integral(100,500); // ritorna i conteggi tra il bin 100 e 500
	h->Integral(100,500,"width"); // integra tra 100 e 500
	
\end{lstlisting}
Il primo ritorna l'integrale di tutto l'istogramma, il secondo invece integra solo la parte compresa tra i bin 100 e 500 compresi. Attenzione, solo Integral(a,b) non fa l'intrale in senso comune, non calcola l'area dell'istogramma, ma ritorna solo in numero di conteggi comrpresi tra quei bin, per avere l'area bisogna aggiungere l'opzione width.  Nota bene, negli argomenti non si specifica il range sull'asse x, ma il numero del bin. Inoltre ricorda di tenere in considerazione il fatto che i bin contengono solo il loro estremo inferiore, per evitare di integrare erroneamente gli estremi. Qualora usassi la funzione Find infatti per integrare sull'asse x da 0 a 10 bisogna usare gli estremi Find(0) e Find(10*0.9999999), poichè Find(10) corrisponde al bin che contiene 10 come estremo inferiore.\\
Per normalizzare un istrogamma invece si può usare la funzione
\begin{lstlisting}[language=C++,label={cod1},mathescape=true,breaklines=true]
	h->Scale(d); //d e' un double
\end{lstlisting}
in questo modo tutti i bin verranno moltiplicato per d, per capirci se si vuole normalizzare a 1 allora d=1./h-\textgreater Integral() ecc...\\
Quando un istogramma viene stampato con Draw in automatico viene creato un box con le informazioni principali. È possibile modificarlo facendo stampare valori molto utili sia riguardo l'istogramma che riguardo eventiali fit. Il tipo di questo oggetto è TPaveStats, vedere come funziona al segunete \href{https://root.cern.ch/doc/master/classTPaveStats.html}{link} TODO
\section{TGraph}

\subsection{TGraphErrors}

\section{TF1}
Per definire una funzione uno dimensionale definita in un certo range può essere utilizzata la classa TF1.
\begin{lstlisting}[language=C++]
	TF1* f=new TF1("funzione","3*x+2",-10,10);
\end{lstlisting}
vediamo i parametri uno per uno
\begin{itemize}
	\item "funzione": nome con cui dietro le quinte root chiama questo oggetto
	\item "3*x+2": tra le virgolette va messa la funzione da utilizzare, ne parleremo più dettagliatamente in seguito.
	\item -10,10 : questo è il range, si possono mettere sia interi che double, non deve essere per forza simmetrico, ed è il range che viene utilizzato poi per stampare la funzione se la si vuole plottare.
\end{itemize}
\subsection{Vari modi per definire la funzione che si vuole usare}
\begin{itemize}
	\item \textbf{FUNZIONE INLNIE:} la funzione può essere definita direttamente dall'utente senza usare funzioni predefinite, in questo modo metodo rientra anche l'utilizzo delle funzione sin() cos() log() log(10) exp() ecc... la sintassi è la seguente:
\begin{lstlisting}[language=C++]
	TF1 *fa1 = new TF1("fa1","sin(x)/x",0,10);
	fa1->Draw();
\end{lstlisting}
	\item \textbf{FUNZIONE INLINE CON PARAMETRI:} in root NON è possibile definire una funzione in questa forma:
\begin{lstlisting}[language=C++]
	Double_t a=3;
	TF1 *fa1 = new TF1("fa1","a*sin(x)/x",0,10);
\end{lstlisting}
	per utilizzare un parametro all'interno di una funzione definita dall'utente è necessario prima dire a root che la funzione contiene un generico parametro [n] e poi impostarne il valore desiderato all'eserno della definizione del TF1.
\begin{lstlisting}[language=C++]
	TF1 *fa = new TF1("fa","[0]*x*sin([1]*x)",-3,3);
	fa->SetParameter(0,value_first_parameter);
	fa->SetParameter(1,value_second_parameter);
\end{lstlisting}
il primo valore del comando SetParameter si riferisce al numero del parametro [0], [1] ecc... il secondo è il valore che si vuole impostare. Questo può essere specificato direttamente lì, es: SetParameter(0,3); oppure puè essere utilizzata una variabile definita esternamente es: SetParameter(0,k);\\
Quella utilizzata precedentemente non è necessariamente la sintassi più compatta o più leggibile per definire i parametri, possono anche essere utilizzate le seguenti funzioni:
\begin{lstlisting}[language=C++]
	TF1 *fa = new TF1("fa","[0]*x+[1]",-3,3);
	fa->SetParameters(3,4);
\end{lstlisting}
la funzione SetParameters permette di impostarli tutti in una sola riga, scritta così associerà 3 a [0] e 4 a [1], se ci fossero stati altri parametri la logica sarebbe stata la stessa.\\
Un'altra cosa utile a rendere il codice più leggibile è rinominare i parametri:
\begin{lstlisting}[language=C++]
	TF1 *fa = new TF1("fa","[0]*x+[1]",-3,3);
	fa->SetParName(0,"m");
	fa->SetParName(1,"q");
	fa->SetParameter("m",3);
	fa->SetParameter("q",4);
\end{lstlisting} 
Anche reimpostare i nome può essere fatto in modo più compatte utilizzando la stessa sintassi utilizzata per reimpostare i parametri con:
\begin{lstlisting}[language=C++]
	fa->SetParNames("m","q");
\end{lstlisting}
	\item \textbf{FUNZIONE DEFINITA ESTERNAMENTE:} La funzione se ha una definizione particolarmente lunga può essere definita esternamente come una normale funzione di C++ per poi essere utilizzate nella definizione del TF1:
\begin{lstlisting}[language=C++]
	Double_t myFunc(double x) { return x+sin(x); }
	....
	TF1 *fa3 = new TF1("fa3","myFunc(x)",-3,5);
	fa3->Draw();
\end{lstlisting}
\end{itemize}
Questi non sono tutti i modi per definire funzioni, molte di quelle con rilevanza fisica sono già definite dentro root nella classe "TMath", se si è interessati è consigliato leggere il manuale, in questa guida non si vuole andare così a fondo.\\
Gli esempio sono stati presi o basati su quelli del manuale di root a questo \href{https://root.cern.ch/doc/master/classTF1.html}{link}, lì ci sono sia maggiori dettagli sui metodi trattati sia altri non menzionati.

\subsection{Valutare un funzione e le sue derivate}
Per valutare un TF1 chiamato f in un punto il comando è:
\begin{lstlisting}[language=C++]
	f->Eval(x);
\end{lstlisting}
Dove x è l'ascissa. Se invece si vuole valutare una delle derivate 1° 2° o 3° il comando è:
\begin{lstlisting}[language=C++]
	f->Derivative(x);
	f->Derivative2(x);
	f->Derivative3(x);
\end{lstlisting}
Se si vuole aumentare la precisione perchè la funzione è particolarmente strana si può aumentare con:
\begin{lstlisting}[language=C++]
	f->Derivative(x,0,k);
\end{lstlisting}
dove 0 è un valore di default che non va modificato e k è il parametro usato nell'algoritmo. Senza spiegre come funziona l'algoritmo che deriva basta sapere che più è piccolo più la derivata sarà precisa, il valore di default è 0.001. Se si vogliono maggiori informazioni sia su come viene calcolata la derivata sia sul significato dei parametri leggere il manuale ufficiale di root per la classe TF1 e cercare la funzione Derivative a questo  \href{https://root.cern.ch/doc/master/classTF1.html#a59c6f24c0b4d4987e46a83ba0b1f26fa}{link}.

\section{TFile}

TO DO:\\
TBrowser\\

Permette di salvare dei file (con estensione .root) che contengono varie informazioni e oggetti. Possono contenere isotogrammi, funzioni, set di dati ecc... Questo per non dover ricaricare ogni volta tutti i dati e rifare tutta l'analisi per avere determinati valori: se costruisco una volta un istogramma e salvo tutto in un TFile la volta dopo quando mi servirà non devo ricaricare i dati, mi basterà prendere quello salvato che contiene già tutto.

\subsection{Scrittura e creazione}
Il costruttore per un TFile è il seguente:
\begin{lstlisting}[language=C++,label={cod1},mathescape=true,breaklines=true]
	TFile* file= new TFile("nome.root","opzione");
	file->cd(); //opzionale
\end{lstlisting}
In questo modo nella cartella in cui si è eseguito root verrà creato il file "nome.root".\\ Vediamo quindi cosa scrivere al posto di "opzione":

\begin{table}[h]
	\begin{tabular}{|c|c|}
		\hline
		opzione	& descrizione \\
		\hline
		NEW  oppure CREATE	& crea il file se non esiste, altrimenti non fa nulla \\
		\hline
		RECREATE	& crea il file, se esiste lo sovrascrive \\
		\hline
		UPDATE	& apre un file esistente, se non lo trova lo crea \\
		\hline
		REED	& apre un file in sola lattura \\
		\hline
	\end{tabular}
\label{tab:TFile_options}
\caption{\centering opzioni comuni (ma non le uniche) per il costruttore di un TFILE}
\end{table}
Ma a cosa serve il comando file-$>$cd()? Tra le varie funzioni che si possono usare in root alcune riguardano i TFile, ma non si riferiscono ad un TFile specifico. Un esempio che vedremo subito è la funzione Write(), questa non sa in quale TFile salvare un oggetto se ce ne sono definiti più di uno, cd() dice a root che da lì in poi a meno che non sia esplicito per qualche motivo quando si lancia un comando o una funzione che si riferiscono ad un TFile si riferiranno al TFile corrente "file".\\
Per salvare i vari oggetti di root questi hanno una funzione chiamata Write, sarà quindi sufficiente, solo dopo aver creato il file di output, chiamare ad esempio:
\begin{lstlisting}[language=C++,label={cod1},mathescape=true,breaklines=true]
	hist->SetNameTitle("nome","titolo"); // se non e' gia' stato fatto
	file->cd(); //se non e' stato gia' fatto
	hist->Write();
\end{lstlisting}
Qui è chiaro a cosa serve il comando cd(), nella funzione Write() non si esplicita in alcun modo in quale TFile scrivere. Utilizzando il comando cd() da lì in poi Write scriverà nel TFile selezionato fino a nuovo ordine.\\
Se si desidera scrivere l'oggetto nel TFile file2 sarà sufficiente chiamare file2-$>$cd() e poi eseguire di nuovo Write().\\
La funzione SetNameTitle invece serve per impostare appunto un nome e un titolo agli oggetti, questi attributi sono quelli che verranno salvati nel TFile e che ci permetteranno di riconoscere l'oggetto "hist" nel TFile in futuro, per questo è caldamente consigliato impostarli sempre (in ogni caso root ne mette di defualt per sicurezza).\\
Le funzioni principali per gestire nome e titolo sono le seguenti:
\begin{lstlisting}[language=C++,label={cod1},mathescape=true,breaklines=true]
	Object *obj;
	obj->SetNameTitle("nome","titolo");
	obj->SetName("nome");
	obj->SetTitle("titolo");
\end{lstlisting}
Root in automatico associa un nome e un titolo agli oggetti definiti, ma questi sono sempre gli stessi per tutti i TH1D tutti i TF1 ecc... quindi salvare oggetti diversi dello stesso tipo con stessi nome e titolo rende molto fastidioso reuperarli, in quanto saranno difficili da distinguere. Ecco perché è buona norma impostarli prima di chiamare Write().\\
Infine una volta finito di salvare sul file questo va chiuso con
\begin{lstlisting}[language=C++,label={cod1},mathescape=true,breaklines=true]
	file->Close();
\end{lstlisting}
{\color{red} \rule{\linewidth}{0.5mm}}
\textcolor{red}{\textbf{Attenzione:}} eseguire Close è fondamentale. Se si esce da root senza chiamare Close() nel 99\% dei casi andrà tutto liscio perchè quando possibile root lo fa in automatico. Se però ad esempio si chiude il terminale su cui si sta eseguendo root Close() non viene chiamata è tutte le modifiche vengono perse irrecuperabilmente.\\
{\color{red} \rule{\linewidth}{0.5mm}}

\subsection{Lettura}
\begin{lstlisting}[language=C++,label={cod1},mathescape=true,breaklines=true]
	TFile* file = new TFile("nome file","opzione"); //crea l'oggetto
	file->cd();
\end{lstlisting}
Questo comando in generale crea l'oggetto file, al posto di "opzione" mettere l'opzione che si desidera tra UPDATE e REED (vedi tabella \ref{tab:TFile_options}) che potrà poi essere utilizzato di seguito. Per sapere però cosa c'è dentro è possibile proseguire in due modi, o si usa un TBrowser (vedremo di seguito cosa è e come si usa), oppure si usa l'interprete da riga di comando e il comando ".ls". Con il TBrowser è possibile nagivare al suo interno come fosse una cartella, con il comando .ls invece a schermo verrà stampato il suo contenuto.\\
".ls" è un altro esempio del perché è necessario chiamare cd(), con .ls non si specifica in alcun modo di che TFile mostrare il contenuto, e a schermo verrà stampato quello del TFile corrente che va impostato in precedeza.\\
A questo punto sappiamo come vedere cosa c'è all'interno di un TFile, vediamo ora come recuperare il suo contenuto per utilizzarlo:
\begin{lstlisting}[language=C++,label={cod1},mathescape=true,breaklines=true]
	TH1D* h = (TH1D*)file->Get("nome oggetto");
\end{lstlisting}
Per questo comando è necessaria una conversione perchè la funzione Get ritorna un TObject*, tutti gli oggetti in root sono una derivata di questa classe, di conseguenza così va sul sicuro, tu però poiché sai il tipo effettivo dell'oggetto che stai copiando puoi direttemante convertire il puntatore al tipo corretto (in questo caso un TH1D*).
Questo metodo è il più esplicito ma in realtà è possibile lasciar fare tutto a root in background, se nel TFile c'è un istogramma chiamato "h" dopo aver creato il TFile e aver eseguito il relativo cd() sarà possibile utilizzare l'istogramma semplicemente con il suo nome: sarà possibile eseguire h-$>$Draw() h-$>$Fill() ecc... lasciando fare in background a root tutte le operazioni di copia.
{\color{red} \rule{\linewidth}{0.5mm}}
\textcolor{red}{\textbf{Attenzione:}} Questo metodo è un po' fuorviante, se ad esempio da un TFile si lascia fare tutto a root in background sarebbe naturale pensare che chiamando ad esempio h-$>$Fill(3) nel TFile l'istogramma abbia un conteggio in più nel bin che contiene 3, ma così non è. Non si sta modificando l'oggetto contenuto nel TFile ma una sua copia con lo stesso nome. Per salvare eventuali modifiche bisogna ricordare si chiamare h-$>$Write()\\
{\color{red} \rule{\linewidth}{0.5mm}}

\section{Gestione oggetti all'interno di un TFile}
Il TFile non va interpretato come una "cartella" di un file manager in cui gli oggetti si possono rinominare o modificare direttamente. È più una "foto" degli elementi che contiene. Questi elementi possono essere copiati per essere utilizzati altrove o eliminati, ma non possono essere direttamente modificati. Vediamo per passi come gestire un TFile:\\
\textbf{Inserimento}: per aggiungere un oggetto basta seguire la procedura descritta precedentemente, questo nel file sarà riconducibile ad un nome e un titolo.\\

\textbf{Salvare più volte oggetti con lo stesso nome:} Questa situazione può presentarsi in due occasioni: nel TFile c'è una versione precedente di un oggetto, questo è stato copiato, modificato, e ora va ri-salvato senza modificare nome e titolo, chiamiamo questa situazione "aggiornamento", oppure non sono stati impostati nome e titolo personalizzati per due oggetti distinti (esempio due istogrammi) e salvandoli nel TFile sembra che lo stesso oggetto sia stato salvato due volte (con nome e titolo di defaul), chiamiamo questa situazione "errore nomi". Vediamo prima come root gestisce i salvataggi nel TFile e poi valutiamo le due situazioni precedenti.\\
I TFile in automatico non sovrascrivono gli oggetti, qualora si salvasse più volte un oggetto con lo stesso nome (ad esempio "istogramma") chiamando .ls nel TFile risulteranno più oggetti "istogramma" con nomi: "istogramma;1" "istogramma;2" e così via. Nel TFile vengono salvate tutte le versioni di un oggetto in modo da poter recuperare le precedenti in caso di errore. Per distinguerli quando si chiama la funzione Get è sufficiente scrivere:
\begin{lstlisting}[language=C++,label={cod1},mathescape=true,breaklines=true]
	TH1D *h = (TH1D *)file->Get("istogramma;1");
	//oppure
	TH1D *h = (TH1D *)file->Get("istogramma;2");
\end{lstlisting}
La chiamata con Get("istogramma") ritornerà in automatico l'ultima versione disponibile, in questo caso la 2. Vediamo ora come gestire le due situazioni iniziali con oggetti di tipo TH1D come esempio:\\
\begin{itemize}
	\item \textbf{Aggiornamento:} dopo aver copiato, e modificato un TH1D di un TFile questo va risalvato con Write(), nel TFile comparità la sua versione "istogramma;2" e tutto è filato liscio. A questo punto si può eliminare la versione precedente oppure lasciare tutto così come è.
	\item  \textbf{errore nomi:} per sbaglio due oggetti distinti di tipo TH1D con nome di default sono stati salvati in un TFile, all'interno di questo, poichè hanno stesso nome risultano essere due versioni di uno stesso oggetto: "istogramma;1" e "istogramma;2". Non c'è modo di rinominare direttamente uno dei due per renderli distinguibili, per risolvere il problema bisogna fare quanto segue:
\begin{lstlisting}[language=C++,label={cod1},mathescape=true,breaklines=true]
TH1D *h = (TH1D *)file->Get("istogramma;1");
h->SetName("nome personalizzato");
h->Write();
file->Delete("istogramma;1");
\end{lstlisting}
in questo modo si copia l'oggetto, gli si da un nome personalizzato, lo si salva di nuovo nel file, e si elimina il corrispettivo con quello di defualt. Questa procedura va fatta almeno per uno dei due così da distinguerli, poi è consigliato farla su entrambi per maggiore chiarezza.\\
{\color{blue} \rule{\linewidth}{0.5mm}}
\textcolor{blue}{\textbf{Nota:}} Il titolo non concorre a questo problema di riconoscimento, se due oggetti hanno lo stesso nome e titolo diversi comunque nel TFile verranno create le versioni 1 e 2, ma almeno all'utente a prima vista queste saranno distinguibili tramite il titolo. Se invece anche questo è lo stesso tra le versione dell'oggetto l'unico modo per distinguerle è vedere cosa c'è dentro\\
{\color{blue} \rule{\linewidth}{0.5mm}}
\end{itemize}

\textbf{Cancellare oggetti in un TFile:} per questo esiste la funzione Delete(). Vediamo qualche esempio:
\begin{lstlisting}[language=C++,label={cod1},mathescape=true,breaklines=true]
	TFile *f=new TFile("file.root","UPDATE");
	.....
	f->Delete("nome"); //cancella l'oggetto "nome"
	f->Delete("nome;1") //cancella la versione 1 di "nome"
	f->Delete("*;1") //cancella la versione 1 di tutti gli oggetti
	f->Delete("nome;*") // cancella tutte le versioni di "nome"
\end{lstlisting}
Inoltre quando in un TFile sono presenti più versioni di un oggetto è necessario spcificare quale/i versioni si desidera eliminare, chiamare un Delete("nome") generico non avrà alcun effetto.
\section{TNtuple}
Questo oggetto risulta particolarmente utile per la gestione dei dati da analizzare. Serve per memorizzare in modo efficiente e pratico dei set di dati, come potrebbero essere delle cordinate spaziali (x,y,z) per un numero n (non necessariamnete noto) di eventi. Si può così evitare di dover definire 3 vettori per le varie coordinate, potendo gestire tutti i dati con un solo oggetto.\\
La limitazione più importante è che i TNtuple sono degli oggetti "write-once, read many times", di conseguenza un TNtuple non può essere modificato, l'unico modo per "modificarlo" è copiarlo in uno nuovo modificandolo come si vuole. Altra sottigliezza invece è il tipo dei dati da registare, per le TNtuple i valori devono essere dei float (Float\_t per la precizione), si può avere maggere precisione utilizzando i TNtupleD con dei Double\_t, ma in genereale non si possono salvare elementi con un tipo arbitrario. Per questo vedremo i TTree successivamente.\\
Vediamo ora il costruttore di questo oggetto.
\begin{lstlisting}[language=C++,label={cod1},mathescape=true,breaklines=true]
	TNtuple *nt = new TNtuple("ntName","titolo","var1:var2:var3");
\end{lstlisting}
I primi due argomenti sono intuitivi e non necessitano di spiegazioni, per la lista delle variabili invece: non ce nè un numero massimo o minimo, nell'esempio del costruttore ce ne sono tre per caso, potevano essere due, quattro ecc... Invece di procedere in casi generali vediamo attraverso un esempio il loro utilizzo.


\subsection{Inserimento e recupero dati}
\begin{lstlisting}[language=C++,label={cod1},mathescape=true,breaklines=true]
	TNtuple *nt = new TNtuple("event_decay","posizione decadimenti","n_eve:x:y:z"); //costruttore
\end{lstlisting}
I principali metodi di inserimento sono due; evento per evento attraverso la funzione Fill
\begin{lstlisting}[language=C++,label={cod1},mathescape=true,breaklines=true]
	nt->Fill(0,0.3,0.1,-0.25); // inserisce un set di valori
\end{lstlisting}
Oppure leggendo un file di input
\begin{lstlisting}[language=C++,label={cod1},mathescape=true,breaklines=true]
	nt->ReadFile("nomefile");
\end{lstlisting}
Il file in questo caso dovrà essere nel formato:\\
n\_eve x y z\\
\# commento\\
n\_eve x y z\\
.....\\
La lettura ignorerà le righe che inizano per \# interpretandole come commenti
È poi possibile recuperare i dati inseriti nel modo seguente:
\begin{lstlisting}[language=C++,label={cod1},mathescape=true,breaklines=true]
	float ex,ey,ez;
	nt->SetBranchAddress("x",&ex);
	nt->SetBranchAddress("y",&ey);
	nt->SetBranchAddress("z",&ez);
	nt->GetEntry(2);
\end{lstlisting}
Con il comando SetBranchAddress come dice il nome lega i float indicati alla variabile x, y, z della TNtuple. Utilizzando poi GetEntry(2) "carica" i valori dell'evento 2 in ex, ey, ez. 

\subsection{Stampare a schermo informazioni utili}
Fino ad ora abbiamo visto comandi utili da inserire in un .cc da compilare per tirarne fuori un eseguibile,  vediamo ora alcune funzioni utili da utilizzare nell'interprete da terminale dopo aver caricato una macro che contiene una TNtuple.
\begin{lstlisting}[language=C++,label={cod1},mathescape=true,breaklines=true]
	nt->Print(); // stampa le informazioni sul contenuto
	nt->Scan();  // stampa tutto il contenuto
	nt->Scan("x") // stampa tutti i valori di x
	nt->GetEntries() //stampa il numero di eventi inseriti
\end{lstlisting}
È poi possibile accedere ai vari eventi sempre stampandoli sullo schermo in base alla "posizione" all'interno del TNtuple. Il primo inserito è lo 0, il secondo è l'1 e così via:
\begin{lstlisting}[language=C++,label={cod1},mathescape=true,breaklines=true]
	nt->Show(0); //stampa l'evento 0
\end{lstlisting}

\subsection{Ricavare grafici da un TNtuple}
Avendo al suo interno una grande quantità di dati sono state scritte delle funzioni che permettono di stampare vari grafici direttamente da un TNtuple:
\begin{lstlisting}[language=C++,label={cod1},mathescape=true,breaklines=true]
	nt->Draw("varexp","selection");  //generale
	nt->Draw("x") // stampa un istogramma dei valori di x
	nt->Draw("x*y") // stampa un istogramma dei valori di x*y
	nt->Draw("x","x>0") // stampa un istogramma dei valori di x per cui x>0
	nt->Draw("x:y") // stampa un grafico cartesiano (x,y)
	nt->Draw("x:y","x>0 & y<0") // come sopra con le condizioni x>0 && y<0
\end{lstlisting}
Come si capisce dagli esempi qualora non venga immessa alcuna selezione verranno stampati tutti gli elementi. Inoltre è possibile stampare non solo le variabili ma anche funzioni di esse.\\
È chiaro che nello stampare i dati sia stato creato un istogramma o un TGraph temporaneo, è possibile però riempire un TGraph o un TH1 definito dall'utente in precedenza utilizzando il comando:
\begin{lstlisting}[language=C++,label={cod1},mathescape=true,breaklines=true]
	nt->Project("h","varexp","selection");
\end{lstlisting}
il primo argomento è l'oggetto che riempie: un TH1, TH2 o un TH3, gli argomenti successivi sono analoghi al precedente Draw(). Ve evidenziata però una differenza tra l'utilizzo di questa funzione da linea di comando o da una macro. Da linea di comando l'oggetto h non deve essere creato necessariamente in anticipo, verrà creato al momento in automatico e h sarà il puntatore all'itogramma in questione. In tal caso viene creato con un nome, un titolo, un range e un binning automatici. Se invece si preferisce specificarli bisogna creare h prima di eseguire Project.\\
Se si usa questa funzione in una macro l'oggetto sarà comunque creato in automatico, ma nella macro non sarà possibile utilizzarlo come puntatore. Nel caricare il file contenente la macro root ritorna un errore in cui dice che h non è stato ancora definito. Se si vuole quindi usare l'istogramma h all'interno della stessa macro in cui lo si crea attraverso Project h va dichiarato prima di chiamare Project con il costruttore regolare, la funzione Project invece di crearlo e riempirlo riempirà quello già creato.\\
Altra nota è che in questo modo non possono essere estratti dei TGraph, per farlo bisogna usare un metodo più macchinoso:
\begin{lstlisting}[language=C++,label={cod1},mathescape=true,breaklines=true]
	nt->Draw("py:px","pz>4");
	TGraph *gr = new TGraph(nt->GetSelectedRows(), nt->GetV2(), nt->GetV1());
\end{lstlisting}
Per non scendere nei dettagli (che non sappiamo nemmeno noi) basta sapere che prima è necessario lanciare Draw("var1:var2:var3:...") con all'interno lisate le variabili che vuoi plottare. in questo modo viene salvata la selezione, poi poi bisogna costruire il TGraph.\\

\section{TTree}
I TTree sono oggetti simili agli TNtuple, ma più generali, permettendo di mettere in memoria oggetti di un tipo qualunque.
Come suggerisce il nome sono costruiti in analogia con un albero. L'albero generale (TTree) è l'oggetto che contiene tutte le informazioni, queste si dividono in rami (TBranch) che a loro volta possono contenere più elementi dello stesso tipo (che siano vettori, strutture o double). I singoli elementi all'interno di un TBranch sono detti leaves (foglie). Un albero puo avere più rami, uno che magari contiene dei vettori, uno che contiene strutture e così via.\\
Ecco perché i TTree sono la generalizzazione degli TNtuple, la logica è la stessa ma può essere applicata a oggetti di vario tipo.\\
Vediamo adesso il loro costruttore:
\begin{lstlisting}[language=C++,label={cod1},mathescape=true,breaklines=true]
	TTree *tree = new TTree("treename","titolo");
\end{lstlisting}
A questo punto per iniziare a riempirlo è necessario creare un branch. Per rimanere più generali possibile immaginiamo di voler mettere in memoria degli oggetti di tipo Object, che siano vector, strutture, classi personalizzate, non è rilevante ora. La sintassi per salvare oggetti nel TTree è la seguente:
\begin{lstlisting}[language=C++,label={cod1},mathescape=true,breaklines=true]
	Object ob;
	TBranch *branch = tree->Branch("branchname",&ob,"titolo");
\end{lstlisting}
Il branch è stato legato all'oggetto ob, a questo punto dovrò utilizzare ob per inserire all'interno del branch le varie foglie (il tipo delle folgie è implicitamente definito da ob). Inserisco dentro ob tutto quello che mi serve per quel relativo oggetto e chiamo:
\begin{lstlisting}[language=C++,label={cod1},mathescape=true,breaklines=true]
	br->Fill();
\end{lstlisting}
Ora per inserire un'altra foglia dovrò ricaricare \underline{sempre dentro ob} le informazioni che mi servono e richiamare Fill(). Si Creeranno così le varia foglie dentro il branch, nominate con l'indice 0, 1, 2 ecc...

\subsection{Recupero dati salvati in un TTree}
La scrittura  e il recupero di un tree all'interno di un TFile sono analoghi a quelli descritti nella sezione apposita, di conseguenza non verranno trattati nuovamente. Vediamo invece più nel dettaglio come estraare i dati dal tree.
\begin{lstlisting}[language=C++,label={cod1},mathescape=true,breaklines=true]
	TBranch *br = tree->GetBranch("branchname"); //Seleziono il banch che mi interessa
	Object ob;
	br->SetAddress(&ob)
	br->GetEntry(0)
\end{lstlisting}
La logica è analoga a quella per le TNtuple, all'interno di un branch ho molti oggetti (che immaginiamo essere le leaves). Poichè il tree verosimilmente avrà più branch per prima cosa devo sceglierne uno mettendo in memoria il suo puntatore. A Questo punto lego una variabili al branch e attraverso GetEntry(0) carico la prima leaf nell'oggetto ob, se avessi usato GetEntry(1) avrei caricato la seconda caricata ecc...