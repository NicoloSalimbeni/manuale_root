\chapter{FIT}
\section{Introduzione generale}
In questa sezione vediamo come fare dei fit, la logica è la stessa per tutti gli oggetti che hanno una funzione Fit, di seguito si farà riferimento ad un istogramma generico h come oggetto da fittare, ma questo potrebbe essere benissimo anche un TGraph o un TGraphErrors ad esempio (questo non è del tutto vero se si vuole approfondire l'argomento, ma le opzioni base, che sono le sole riportate, sono comuni a tutti gli oggetti che ci interessano).\\
La funzione generale per eseguire un fit è nella fomra:
\begin{lstlisting}[language=C++,label={cod1},mathescape=true,breaklines=true]
h->Fit (TF1 * f1,
Option_t *  	option = "",
Option_t *  	goption = "",
Double_t  	xmin = 0,
Double_t  	xmax = 0 
) 	
\end{lstlisting}
Dove f1 è la funzione con cui eseguire il fit, option è una stringa che specifica le opzioni del fit, goption una stringa con le opzioni grafiche, mentre xmin e xmax sono il reange entro cui viene eseguito il fit. Vediamo pezzo per pezzo le opzioni più rilevanti.\\
La funzione per eseguire il fit base dell'oggetto h attraverso il TF1 f è:
\begin{lstlisting}[language=C++,label={cod1},mathescape=true,breaklines=true]
	h->Fitt(f);
\end{lstlisting}
Se ad esempio si ha intenzione di impostare un range senza fornire alcuna option e goption la sintassi deve essere le seguente:
\begin{lstlisting}[language=C++,label={cod1},mathescape=true,breaklines=true]
	h->Fitt(f,"","",xmin,xmax);
\end{lstlisting}
Da cui è chiaro come fare se si vuole lasicare una o più opzioni di default.\\
L'algoritmo di defaul per trovare i parametri ricercati è la minimizzazione del $\chi^2$. Il range predefinito è quello di definizione dell'istogramma. Ultima cosa a cui presare attenzione è che in questo modo verrà memorizzato solo l'ultimo fit eseguito rimuovendo i precedenti, se quindi si volessero fare più fit su uno stesso grafico sarà necessario cambiare le opzioni manualmente (vedi paragrafo successivo).\\
Inoltre la TF1 fornita per eseguire il fit deve avere dei parametri nella forma [0] [1] ecc... come mostrato nel paragrafo \ref{TF1_section} alla voce "funzione inline con parametri"

\section{Fit options}
La sintassi per inserire opzioni aggiuntive nel fit è la seguente:
\begin{lstlisting}[language=C++,label={cod1},mathescape=true,breaklines=true]
	h->Fit(f,"opzione")
\end{lstlisting}
Nell'esempio si sottointende che si vuole fittare un generico oggetto h, che sia un istogramma, un grafico ecc... con una generica funzione f. Vediamo le opzioni che possono tornare utili più spesso:
\begin{itemize}
	\item "R" : prende il range di definizione del TF1, fa il fit solo per i punti in quell'intervallo, e stampa il grafico in quell'intervallo
\begin{lstlisting}[language=C++,label={cod1},mathescape=true,breaklines=true]
	h->Fit(f,"R");
\end{lstlisting}
	\item "+" : se sono stati fatti fit precedenti, solitamente viene stampato solo l'ultimo, se invece metti il + allora questo fit non sovrascive i precedenti e nel plot lì troverai tutti
\begin{lstlisting}[language=C++,label={cod1},mathescape=true,breaklines=true]
h->Fit(f,"+");
\end{lstlisting}
	\item "0" : se vuoi fare il fit, ma vuoi evitare di stamparlo sul plot
\begin{lstlisting}[language=C++,label={cod1},mathescape=true,breaklines=true]
	h->Fit(f,"0");
\end{lstlisting}
	\item "S" : nel ritornare i risultati (vedi gestione risultati) nel TFitResult vengono salvate tutti le possibili informazioni, tra cui la matrice di covarianza, alla quale normalmente non sarebbe possibile accedere
\begin{lstlisting}[language=C++,label={cod1},mathescape=true,breaklines=true]
	h->Fit(f,"S");
\end{lstlisting}
\end{itemize}
Se si volesse usare più di una opzione è sufficiente metterle in fila come segue:
\begin{lstlisting}[language=C++,label={cod1},mathescape=true,breaklines=true]
	h->Fit(f,"RS+");
\end{lstlisting}

\subsection{Inserimento parametri e accortezze varie}
L'algoritmo di minimizzazione non è infallibile, specilamente per funzioni particolarmente strane. Ciò significa che se lo si esegue senza dargli alcuna indicazione è molto probabile che dia risultati sbagliati. Per evitare il problema quando possibile è bene fornire dei dati iniziali da cui farlo partire. Per fare ciò si può usare la funzione SetParameter() (o simili), che è stata mostrata nel paragrafo \ref{TF1_section}. I dati iniziali non hanno nulla a che fare con C++ o root ma vanno determinati in base alla situazione fisica che si sta studiando.
{\color{red} \rule{\linewidth}{0.5mm}}
\textcolor{red}{\textbf{Attenzione:}} Spesso anche se il fit è errato il $\chi^2$ da valori accettabili, e i parametri trovati possono differire anche solo di pochi ordini di grandezza da quelli corretti, quindi per funzioni strane prestare particolre attenzione ai dati iniziali forniti all'algoritmo   \\ 
{\color{red} \rule{\linewidth}{0.5mm}}
Per dare ulteriori informazioni su cui basarsi a root è possibile utilizzare le seguenti funzioni:
\begin{lstlisting}[language=C++,label={cod1},mathescape=true,breaklines=true]
	f->SetParameter(0,5); //valore iniziale parametro 0 a 5
	f->SetParLimits(0,-3,3); //parametro 0 tra -3 3
	f->FixParameter(0,4); //fissa il parametro 0 a 4
\end{lstlisting}

\subsection{Gestione dei risultati}
La funzione Fit oltre a fare il fit ritorna anche un oggetto specifico per la gestione dei risultati di quest'ultimo (TFitResultPtr, che è essenzialmente un puntatore ad un oggetto di tipo TFitResult). Il modo più rapido quindi per recuperare tutte le informazioni rilevanti è il seguente:
\begin{lstlisting}[language=C++,label={cod1},mathescape=true,breaklines=true]
	TFitResultPtr r = h->Fit(myFunc,"S");
	TMatrixDSym cov = r->GetCovarianceMatrix();  //  matrice di covarianza
	Double_t chi2 = r->Chi2(); // chi2
	Double_t par0 = r->Parameter(0); // ritorna il parametro 0
	Double_t err0 = r->ParError(0); // ritorna l'errore del parametro 0
	r->Print("V");     // stampa tutte le informazioni possibili
	r->Write();        // salva i risultati in un file
\end{lstlisting}
L'oggetto r potrà poi essere salvato in un TFile, i fit a livello computazionale possono essere molto onerosi, in questo modo se in futuro si dovessero recuperare i risultati di un fit invece di rieseguirlo sarà suffciente leggerli all'interno del TFitResult salvato nel TFile.
\section{Tips}
\textbf{Fare il fit in un range, poi stampare f in un altro range:} la cosa più comoda è prima fare come segue:
\begin{lstlisting}[language=C++,label={cod1},mathescape=true,breaklines=true]
	h->Fit(f,"0","",xmin,xmax);
	f->SetRange(xinf,xsup);
	f->Draw();
\end{lstlisting}
in questo modo si calcola f, si sceglie il range desiderato per il plot e poi la si stampa.\\
\textbf{mostrare i risultati di un fit direttamente sul grafico:} inserire nel codice della macro:
\begin{lstlisting}[language=C++,label={cod1},mathescape=true,breaklines=true]
	gStyle->SetOptFit(1);
\end{lstlisting}